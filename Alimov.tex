\documentclass[a4paper]{article}
\usepackage[14pt]{extsizes} % для того чтобы задать нестандартный 14-ый размер шрифта
\usepackage[utf8]{inputenc}
\usepackage[english,russian]{babel}
\usepackage{indentfirst}
\usepackage{setspace,amsmath}
\usepackage[left=20mm, top=15mm, right=15mm, bottom=15mm, nohead, footskip=10mm]{geometry} % настройки полей документа
\usepackage{graphicx}
\graphicspath{{noiseimages/}}
 
\begin{document} % начало документа
 
% НАЧАЛО ТИТУЛЬНОГО ЛИСТА
\begin{center}
\hfill \break
\large{ГУАП}\\
\hfill \break
\normalsize{Кафедра № 63}\\
\hfill\break
\hfill \break
\begin{flushleft}
\textsc{отчёт}\\
\textsc{защищён с оценкой}\\
\textsc{преподаватель}\\
\end{flushleft}
\renewcommand{\arraystretch}{1} %% increase table row spacing
\renewcommand{\tabcolsep}{1cm}
\begin{center}
\begin{tabular}{ccc}
ст.преп. &  & М.Д. Поляк\\
\hline
\small{должность, уч. степень, звание} & \small{подпись, дата} & \small{инициалы, фамилия} \\
\hline
\end{tabular}
\end{center}
\hfill \break
\hfill \break
\textsc{отчёт о курсовой работе}\\
\hfill \break
\hfill \break
\hfill \break
\large{Добавление защиты от несанкционированного запуска операционной системы.}\\
\hfill \break
\hfill \break
\hfill \break
по \textsc{операционным системам}\\
\hfill \break
\hfill \break
\hfill \break
\begin{flushleft}
\textsc{работу выполнил:}
\end{flushleft}
\renewcommand{\arraystretch}{1} %% increase table row spacing
\renewcommand{\tabcolsep}{0,6cm}
\begin{tabular}{cccc}
\textsc{студент гр.} & 4331 & & А.С. Алимов \\
\hline
& & \small{подпись, дата} & \small{инициалы, фамилия} \\
\end{tabular}
\end{center}
\hfill \break
\hfill \break
\begin{center} Санкт-Петербург, 2016 \end{center}
\thispagestyle{empty} % выключаем отображение номера для этой страницы
 
% КОНЕЦ ТИТУЛЬНОГО ЛИСТА
 
\newpage

\section{Цель работы}

Знакомство с устройством ядра ОС Linux. Получение опыта разработки драйвера устройства.

\section{Задание}

Необходимо внести изменения в процесс загрузки ядра Linux, добавив проверку наличия
подключенного через интерфейс USB flash-накопителя с заданным серийным номером.
Если в процессе загрузки операционной системы нужный flash-накопитель подключен к
одному из портов USB, то операционная система успешно загружается в штатном режиме.
Если flash-накопитель с нужным серийным номером отсутствует, отобразить на экране
предупреждение и таймер с обратным отсчетом (30 секунд), загрузка операционной
системы при этом приостанавливается. По истечении обратного отсчета таймера должно
происходить автоматическое выключение компьютера. При подключении к любому из
USB-портов нужного flash-накопителя во время обратного отсчета таймера, таймер должен
останавливаться, после чего операционная система должна загружаться в штатном
режиме.

\section{Техническая документация}


Действия, выполняемые при добавлении Linux-драйвера, а также следующие специальные действия с флэш-накопителем:\\

Шаг 1: Собираем драйвер (имя\_файла.ko) с помощью запуска команды make.\\

Шаг 2: Копируем файл в папку /lib/modules/версия ядра/имя\_файла.ko.\\

Шаг 3: Добавляем драйвер в автозагрузку с помощью команды depmod имя\_файла.ko.\\

Шаг 4: Отключаем флеш-устройство при загрузке системы\\

Шаг 5: Перезгружаем систему\\

Шаг 6: Вставляем USB флэш-накопитель с заданным номером чтобы разблокировать систему\\

Скриншот отсчета времени до перезагрузки системы\\

\begin{figure}[h!]
\clearpage
\centering
\includegraphics[width=16cm]{screen.eps} 
\caption{Лог модуля}
\end{figure} 

Выгружаем драйвер с помощью команды rmmod имя\_файла.ko.\\

\section{Вывод}

Проделав работу я ознакомился с ядром ОС Linux и получил опыт разработки драйвера USB устройства.

\section{Приложение}

\begin{verbatim}
#include <linux/kernel.h>
#include <linux/module.h>
#include <linux/usb.h>
#include <linux/sched.h>
#include <linux/kthread.h>
#include <linux/delay.h> 
#include <linux/reboot.h> 

struct task_struct *tAgetty;
struct task_struct *tTimer;
struct task_struct *task;
bool stopThread = true;
bool isTimer = true;

static int thread_timer( void * data) 
{
	int i = 30;

	while(stopThread)
	{
		printk(KERN_ERR "[!!!reboot %i sec!!!] to cancel the insert flash in USB [%d]\n", i, current->pid );
		ssleep( 1 );
		if (i <= 0) kernel_restart(NULL);
		i--;
	}
	return 0;
}

static int thread_agetty_uninterrupyible( void * data) 
{
	// поток таймера
	tTimer = kthread_create( thread_timer, NULL, "thread_sleep" );

	// основной цикл потока
	while(stopThread)
	{
		for_each_process(task)
		{
			if (strcmp(task->comm, "agetty") == 0 && task->state == TASK_INTERRUPTIBLE)
			{
				ssleep(1);
				printk(KERN_ERR "alim_tty: %s [%d] %u \n", task->comm , task->pid, (u32)task->state);
				task->state = TASK_UNINTERRUPTIBLE;
				// запускаем поток таймера
				if (!IS_ERR(tTimer) && isTimer)
				{
					isTimer = false;
					printk(KERN_INFO "alim_thread: %s start\n", tTimer->comm);
					wake_up_process(tTimer);
				}
			}
		}
	}
	return 0;
}

static int pen_probe(struct usb_interface *interface, const struct usb_device_id *id)
{
	stopThread = false;
	printk(KERN_ERR "alim_flash: connect\n");
	
	for_each_process(task)
	{
		if (strcmp(task->comm, "agetty") == 0 && task->state == TASK_UNINTERRUPTIBLE)
		{
			printk(KERN_INFO "alim_flash: %s [%d] %u \n", task->comm , task->pid, (u32)task->state);
			task->state = TASK_INTERRUPTIBLE;
		}
	}

	return 0;
}

static void pen_disconnect(struct usb_interface *interface)
{
	printk(KERN_ERR "alim_flash: disconnect\n");
}

static struct usb_device_id pen_table[] =
{
	{ USB_DEVICE(0x0C76, 0x0005) },
	{ USB_DEVICE(0x8564, 0x1000) },
    {} /* Terminating entry */
};
MODULE_DEVICE_TABLE (usb, pen_table);

static struct usb_driver pen_driver =
{
	.name = "alim_driver",
	.probe = pen_probe,
	.disconnect = pen_disconnect,
	.id_table = pen_table,
};

static int __init pen_init(void)
{
	printk("alim_init: start\n");

	// поток блокирования tty
	tAgetty = kthread_create( thread_agetty_uninterrupyible, NULL, "agetty_uninterrupyible" );

	if (!IS_ERR(tAgetty))
	{
		printk(KERN_INFO "alim_thread: %s start\n", tAgetty->comm);
		wake_up_process(tAgetty);
	}
	else
	{
		printk(KERN_ERR "alim_thread: agetty_uninterrupyible error\n");
		WARN_ON(1);
	}

	return usb_register(&pen_driver);
}

static void __exit pen_exit(void)
{
	usb_deregister(&pen_driver);
}

module_init(pen_init);
module_exit(pen_exit);

MODULE_LICENSE("GPL");
MODULE_AUTHOR("Alimov Alexandr Sergeevich<a1imov233@gmail.com>");
MODULE_DESCRIPTION("USB Driver");
\end{verbatim}

\bibliography{cours} 
\bibliographystyle{ieeetr}

\end{document}  % КОНЕЦ ДОКУМЕНТА !